%-----------------------------------------------------------------------------------------------------------------------------------------------%
%The MIT License (MIT)
%
%Copyright (c) 2021 Jitin Nair
%
%Permission is hereby granted, free of charge, to any person obtaining a copy
%of this software and associated documentation files (the "Software"), to deal
%in the Software without restriction, including without limitation the rights
%to use, copy, modify, merge, publish, distribute, sublicense, and/or sell
%copies of the Software, and to permit persons to whom the Software is
%furnished to do so, subject to the following conditions:
%
%THE SOFTWARE IS PROVIDED "AS IS", WITHOUT WARRANTY OF ANY KIND, EXPRESS OR
%IMPLIED, INCLUDING BUT NOT LIMITED TO THE WARRANTIES OF MERCHANTABILITY,
%FITNESS FOR A PARTICULAR PURPOSE AND NONINFRINGEMENT. IN NO EVENT SHALL THE
%AUTHORS OR COPYRIGHT HOLDERS BE LIABLE FOR ANY CLAIM, DAMAGES OR OTHER
%LIABILITY, WHETHER IN AN ACTION OF CONTRACT, TORT OR OTHERWISE, ARISING FROM,
%OUT OF OR IN CONNECTION WITH THE SOFTWARE OR THE USE OR OTHER DEALINGS IN
%THE SOFTWARE.
%
%
%-----------------------------------------------------------------------------------------------------------------------------------------------%

%----------------------------------------------------------------------------------------
%DOCUMENT DEFINITION
%----------------------------------------------------------------------------------------

% article class because we want to fully customize the page and not use a cv template
\documentclass[a4paper,10pt]{article}

%----------------------------------------------------------------------------------------
%FONT
%----------------------------------------------------------------------------------------

% % fontspec allows you to use TTF/OTF fonts directly
% \usepackage{fontspec}
% \defaultfontfeatures{Ligatures=TeX}

% % modified for ShareLaTeX use
% \setmainfont[
% SmallCapsFont = Fontin-SmallCaps.otf,
% BoldFont = Fontin-Bold.otf,
% ItalicFont = Fontin-Italic.otf
% ]
% {Fontin.otf}

%----------------------------------------------------------------------------------------
%PACKAGES
%----------------------------------------------------------------------------------------
\usepackage{url}
\usepackage{parskip} 

%other packages for formatting
\RequirePackage{color}
\RequirePackage{graphicx}
\usepackage[usenames,dvipsnames]{xcolor}
\usepackage[scale=0.92, top=1.2cm, bottom=1.3cm]{geometry}
% give TeX room to avoid overflow
\usepackage{tabularx}
\usepackage{enumitem}
\usepackage{supertabular}
\usepackage{titlesec}
\usepackage{multicol}
\usepackage{multirow}
\usepackage[unicode]{hyperref}
\definecolor{linkcolour}{rgb}{0,0.2,0.6}
\hypersetup{colorlinks,breaklinks,urlcolor=linkcolour,linkcolor=linkcolour}
\usepackage{fontawesome5}
%for lists within experience section

\usepackage[UKenglish]{babel}
\usepackage{microtype}               % smarter line breaking
\usepackage{enumitem}                % control list spacing/indent
\setlist[itemize]{leftmargin=*, labelsep=0.35em, itemsep=0.2em, topsep=0.2em}
\setlength{\emergencystretch}{3em}   % give TeX room to avoid overflow
% centered version of 'X' col. type
\newcolumntype{C}{>{\centering\arraybackslash}X} 

%to prevent spillover of tabular into next pages
\usepackage{supertabular}
\usepackage{tabularx}
\newlength{\fullcollw}
\setlength{\fullcollw}{0.47\textwidth}

%custom \section
\usepackage{titlesec}
\usepackage{multicol}
\usepackage{multirow}

%CV Sections inspired by: 
%http://stefano.italians.nl/archives/26
\titleformat{\section}{\Large\scshape\raggedright}{}{0em}{}[\titlerule]
\titlespacing{\section}{0pt}{10pt}{10pt}

%for publications
\usepackage[style=authoryear,sorting=ynt, maxbibnames=2]{biblatex}

%Setup hyperref package, and colours for links
\usepackage[unicode, draft=false]{hyperref}
\addbibresource{citations.bib}
\setlength\bibitemsep{1em}

%for social icons
\usepackage{fontawesome5}

%debug page outer frames
%\usepackage{showframe}


% ---------------- job listing environments (fixed) ----------------
\newenvironment{jobshort}[2]{%
  \noindent
  \begin{tabularx}{\textwidth}{@{}Xr@{}}
    \textbf{#1} & \textit{#2} \\
  \end{tabularx}\par\vspace{-2pt}%
}{%
  \vspace{2pt}%
}

\newenvironment{joblong}[2]{%
  \noindent
  \begin{tabularx}{\textwidth}{@{}Xr@{}}
    \textbf{#1} & \textit{#2} \\
  \end{tabularx}\par\vspace{-4pt}%
  \begin{itemize}[leftmargin=1.1em, itemsep=3pt, topsep=2pt, label=--]%
  \raggedright
}{%
  \end{itemize}\vspace{2pt}%
}
% ------------------------------------------------------------------



%----------------------------------------------------------------------------------------
%BEGIN DOCUMENT
%----------------------------------------------------------------------------------------
\begin{document}

% non-numbered pages
\pagestyle{empty} 

%----------------------------------------------------------------------------------------
%TITLE
%----------------------------------------------------------------------------------------

% \begin{tabularx}{\linewidth}{ @{}X X@{} }
% \huge{Your Name}\vspace{2pt} & \hfill \emoji{incoming-envelope} email@email.com \\
% \raisebox{-0.05\height}\faGithub\ username \ | \
% \raisebox{-0.00\height}\faLinkedin\ username \ | \ \raisebox{-0.05\height}\faGlobe \ mysite.com& \hfill \emoji{calling} number
% \end{tabularx}

\begin{tabularx}{\linewidth}{@{} C @{}}
\Huge{Tanisha Gupta} \\[7.5pt]
\href{https://github.com/Tani843}{\raisebox{-0.05\height}\faGithub\ GitHub} \ $|$ \ 
\href{https://www.linkedin.com/in/tanishagupta008/?skipRedirect=true}{\raisebox{-0.05\height}\faLinkedin\ Linkedin} \ $|$ \ 
\href{mailto:tanishagupta008@gmail.com}{\raisebox{-0.05\height}\faEnvelope \ tanishagupta008@gmail.com} \ $|$ \ 
\href{tel:+000000000000}{\raisebox{-0.05\height}\faMobile \ +44 7920596811}\\
\end{tabularx}

%----------------------------------------------------------------------------------------
% EXPERIENCE SECTIONS
%----------------------------------------------------------------------------------------

%Interests/ Keywords/ Summary
\section*{Professional Summary}

Graduate researcher in \textbf{Applied Mathematics} with research experience spanning 
\textbf{numerical analysis}, \textbf{Bayesian inverse problems}, and 
\textbf{high-performance scientific computing}. I have developed 
\textbf{PAC-Bayesian generalization frameworks} for \textbf{inverse PDE problems}, establishing 
\textbf{finite-sample} and \textbf{mesh-robust uncertainty guarantees} for the 
\emph{inverse heat equation}. My technical portfolio includes 
\textbf{mixed-precision multigrid solvers} demonstrating 
\textbf{GPU-accelerated speedups} and \textbf{second-order convergence}, as well as 
\textbf{Hessian-aware MCMC samplers} for efficient exploration of 
\textbf{high-dimensional posterior distributions}. I aim to build 
\textbf{mathematically rigorous} and \textbf{computationally efficient} methodologies at the 
interface of \textbf{PDE theory}, \textbf{statistical learning}, and 
\textbf{uncertainty quantification}, with applications to 
\textbf{scientific machine learning} and complex \textbf{inverse modelling}.

\section*{Research Interests}

\begin{itemize}
    \item \textbf{Bayesian Inverse Problems \& Uncertainty Quantification:} 
    PAC-Bayesian theory, finite-sample guarantees, mesh-robust bounds, Gibbs/tempered posteriors, 
    certified uncertainty for PDE-governed systems.

    \item \textbf{Numerical Analysis \& Scientific Computing:} 
    Finite-difference and finite-element methods, stability and convergence theory, 
    \textbf{multigrid} and \textbf{Krylov} solvers, mixed-precision numerical computation.

    \item \textbf{High-Dimensional Sampling \& Optimization:} 
    Hessian-aware MCMC, stochastic optimization, adaptive Langevin dynamics, 
    sampling for high-dimensional inverse problems, PDE-constrained optimization.

    \item \textbf{High-Performance Computing for PDEs:} 
    GPU/CUDA acceleration, scalable solver architectures, performance-optimized PDE simulations.

    \item \textbf{Mathematical Foundations of Scientific Machine Learning:} 
    Learning-theoretic guarantees for numerical PDE methods, surrogate modelling with certificates, 
    integration of ML, UQ, and deterministic/stochastic PDE frameworks.
\end{itemize}
%Experience
%-------------------------------------
\section*{Work Experience}

\begin{joblong}{Research Assistant — Mathematics Laboratory, Janki Devi Memorial College (University of Delhi), India}{March 2025 -- Present}
    \item Conducting research on \textbf{Bayesian inverse problems} and \textbf{uncertainty quantification}, focusing on the \textbf{PAC-Bayes framework for PDE-governed systems}.
    \item Developed and validated \textbf{finite-sample generalization bounds} for the inverse heat equation, integrating Bayesian inference with statistical learning theory.
    \item Designed \textbf{mesh-robust numerical solvers} for reliable posterior estimation in high-dimensional inverse problems.
    \item Collaborating with faculty on manuscripts submitted to \textit{SIAM/ASA Journal on Uncertainty Quantification} and related applied mathematics venues.
    \item Mentoring undergraduate students on computational mathematics and statistical inference projects.
\end{joblong}


\begin{joblong}{Junior Research Data Assistant — AI \& Machine Learning Lab, University of Liverpool, UK}{August 2024 -- February 2025}
    \item Researched \textbf{verification frameworks for Deep Reinforcement Learning (DRL)} models, emphasizing algorithmic safety and robustness.
    
    \item Designed \textbf{Lyapunov Barrier Certificate}-based methods reducing instability in DRL controllers by 20\%.
    
    \item Conducted over 50 experiments improving model generalization and convergence behavior.
    
    \item Tuned deep-learning hyperparameters to reduce false positives by 15\% and contributed to peer-reviewed AI journal submissions.
    
    \item \textbf{Tools:} Python (PyTorch, TensorFlow, Scikit-learn), MATLAB, AWS, SQL, Power BI.
\end{joblong}
%-------------------------------------
%-------------------------------------
%---------------------------------------------------------
\section{Projects}

% ---------------------- PROJECT 1 ----------------------
\begin{tabularx}{\linewidth}{ @{}l r@{} }
\textbf{PAC-Bayes Certificates for Bayesian Inverse Problems: A Case Study on the Heat Equation}
& \hfill \href{https://github.com/Tani843/PAC_BAYES_INVERSE_PDE}{GitHub Link} \\[3.75pt]
\multicolumn{2}{@{}X@{}}{
Implements \textbf{PAC-Bayes certified uncertainty} for Bayesian inverse PDEs on the 1D heat equation. 
Provides finite-sample generalization bounds, a mesh-robust decomposition of error, Gibbs/tempered posterior implementation, and a complete, reproducible experiment pipeline (data generation → posterior sampling → certificate computation).
} \\
\end{tabularx}

\vspace{6pt}

% ---------------------- PROJECT 2 ----------------------
\begin{tabularx}{\linewidth}{ @{}l r@{} }
\textbf{Mixed-Precision Multigrid Solvers for PDEs}
& \hfill \href{https://github.com/Tani843/Mixed_Precision_Multigrid_Solvers_for_PDEs}{GitHub Link} \\[3.75pt]
\multicolumn{2}{@{}X@{}}{
High-performance multigrid framework with adaptive mixed precision and CUDA acceleration. 
Demonstrates up to \textbf{6.6× GPU speedup} vs. CPU, \textbf{1.7× mixed-precision gain} with \textbf{35\% lower memory}, and verified \(O(h^2)\) convergence. 
Includes benchmarks, visualization tools, and comprehensive tests.
} \\
\end{tabularx}

\vspace{6pt}

% ---------------------- PROJECT 3 ----------------------
\begin{tabularx}{\linewidth}{ @{}l r@{} }
\textbf{Hessian Aware Sampling in High Dimensions}
& \hfill \href{https://github.com/Tani843/Hessian_Aware_Sampling_in_High_Dimensions}{GitHub Link} \\[3.75pt]
\multicolumn{2}{@{}X@{}}{
Hessian-informed MCMC samplers (Metropolis, Langevin, adaptive variants) for efficient exploration of high-dimensional posteriors. 
Achieves \textbf{2–10× ESS improvements} on ill-conditioned targets, robust to \( d > 10^3 \). 
Ships with benchmarks, diagnostics, and publication-quality plotting utilities.
} \\
\end{tabularx}

%----------------------------------------------------------------------------------------
%EDUCATION
%----------------------------------------------------------------------------------------
%-------------------------------------
\section{Education}
%-------------------------------------

\begin{tabularx}{\textwidth}{@{}l X@{}}

2023 -- 2024 &
\textbf{University of Liverpool, United Kingdom} \\
& Master of Science in Data Science \& Artificial Intelligence \hfill \textbf{Distinction} \\
& \textbf{Relevant Coursework}: Deep Learning, Natural Language Processing, Reinforcement Learning, Big Data, Cloud Computing, Bayesian Statistics, Optimisation Methods, Data Visualisation. \\[6pt]

2019 -- 2022 &
\textbf{Janki Devi Memorial College (University of Delhi), India} \\
& Bachelor of Science (Honours) in Mathematics \hfill \textbf{GPA: 3.6/4.0} \\
& \textbf{Relevant Coursework}: Real Analysis, Differential \& Partial Differential Equations, Complex Analysis, Multivariate Calculus, Linear Algebra, Group Theory, Ring Theory, Numerical Analysis, Probability \& Statistics, Metric Spaces, Optimization, Computational Modelling. \\

\end{tabularx}

%----------------------------------------------------------------------------------------
%PUBLICATIONS
%----------------------------------------------------------------------------------------
\section{Publications}
\begin{enumerate}[leftmargin=*]

\item Tanisha Gupta . 
\textbf{PAC-Bayes Certificates for Bayesian Inverse Problems: A Case Study on the Heat Equation}. 
TechRxiv, July 2025. 
\href{https://doi.org/10.36227/techrxiv.176170993.37005709/v1}{DOI link}. 
(Preprint, under peer review at \textbf{SIAM Journal on Uncertainty Quantification}).

\begin{itemize}
\item \textbf{Methodological novelty:} Introduces the first PAC-Bayesian generalization certificates for Bayesian inverse partial differential equations, combining Gibbs posteriors and tempered Bayesian inference to provide finite-sample, mesh-robust generalization guarantees for inverse-PDE uncertainty quantification.
\end{itemize}

\end{enumerate}

%----------------------------------------------------------------------------------------
%SKILLS
%---------------------------------------------------------------------
\section*{Key Skills}

\textbf{Mathematical \& Statistical Modelling:} Bayesian inference, PAC-Bayesian analysis, 
uncertainty quantification, inverse problems, stochastic optimization, Monte Carlo \& MCMC methods, 
PDE-constrained optimization.

\textbf{Numerical \& Scientific Computing:} Finite-difference \& finite-element methods, 
Crank--Nicolson schemes, multigrid \& Krylov solvers, Hessian-aware sampling, high-performance 
(CUDA/GPU) computing.

\textbf{Programming:} Python (NumPy, SciPy, PyTorch, TensorFlow), C++, MATLAB, R, SQL, Linux, 
Git, CUDA, LaTeX.

\textbf{Machine Learning Foundations:} Statistical learning theory, reinforcement learning 
(safe \& constrained), model interpretability, optimization-based learning.

\textbf{Research Communication:} Technical writing (TechRxiv, SIAM), reproducible GitHub pipelines, 
computational documentation, and academic presentation.

\section*{Certifications \& Technical Achievements}

\textbf{AI \& Machine Learning Specialization (Coursera):} Algorithms, optimization, and model development.\\
\textbf{Data Science with R (SimpliLearn):} Statistical computing and regression modelling.\\
\textbf{Cloud Data Engineering (AWS/GCP):} BigQuery, Spark, and distributed computation.\\
\textbf{AI-Powered Fraud Detection Model:} Designed Python-based ML pipelines reducing false positives by 25\%.\\
\textbf{Advanced Programming Certification (C++ \& Python):} OOP, data structures, algorithms, and applied software development.\\
\end{document}
